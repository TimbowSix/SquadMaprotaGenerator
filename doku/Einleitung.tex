\section{Einleitung}
        % Squad ist ein Online Multiplayerspiel, welches 2015 im \glqq{}Early Access\grqq{} veröffentlicht wurde.\cite{steampage}
        % Für die Umsetzung des Onlinespielens, wird darauf gesetzt, 
        % dass Spielserver von Personen oder Gruppen bereitgestellt und verwaltet werden.
        % \cite{wiki.serverbrowser}
        % Diese Personen/Gruppen sind darum bemüht ihre Server möglichst attraktiv zu halten. \todo{quelle}\\
        % Squad wird in Runden gespielt. Eine Runde besteht aus einem vorherrschenden Spielmodus, 
        % welcher auf einer Kartenvariation (Layer) gespielt wird. 
        % Ein Layer gibt damit an, welcher Spielmodus auf welcher Kartenvariation gespielt wird.\\ \todo{quelle}
        % Um die Attraktivität eines Servers zu steigen, wird versucht eine begehrenswerte Reihenfolge an Runden zu spielen.
        % Dafür wird für jeden Tag eine Reihenfolge an Runden bestimmt, welche auf dem Server gespielt werden sollen.
        % Diese vorgegebene Reihenfolge wird Maprotation oder kurz Maprota genannt.\\
        % Dieses Dokument beschäftigt sich mit der Umsetzung eines Programmes, welches eine attraktive Maprota generieren kann.   

        \subsection{Problemdarstellung}
            Aus den historischen Debatten über die Maprota wurden die Hauptprobleme versucht herauszuheben und im folgenden erklärt.

            Der Mapgenerator soll qualitativ hochwerte Rotas generieren.
            Zur klassifizierung werden wir nun zunächst Eckpunkte definieren welche die Qualität einer Rota bemessen sollen.

            Zunächste sollte sich eine Map nicht zu stark wiederholen.
            Desweiteren gab es in der Community bedenken welche sich damit beschäftigen dass nicht zu ähnliche Maps zu kurz hintereinander gespielt werden sollten. 
            Ein Beispiel für letzteres wäre die Abfolge 
            \begin{equation*}
                \text{Sumari} \rightarrow \text{Logar Valley} \rightarrow \text{Fallujah}.
            \end{equation*}
            Hier würden direkt nacheinander folgend drei relativ ähnliche Maps gezogen werden. 
            Der Charakter dieser Maps wird im wesentlichen über die Eigenschaften "Wüste", "Stadt", "Infanterie lastig/klein" der Map definiert.
            Eine gute Rota sollte solche Map-Ketten vermeiden.

            Ein weitere wichtiger Punkt ist die Vermeidung von Mustern in der Rota. 
            Das bedeutet dass die generierten Rotas nicht deterministisch verteilt sondern aus zufälligem ziehen entstehen sollen.
            
            Das seit einiger Zeit bestehende Layervote-System muss direkten Einfluss auf die Rota haben um die Verteilung den Wünschen der Community anzupassen.

        \subsection{Ziel}
            Zusammenfassend werden wir im folgenden die Qualität der Maprota an folgenden Eigenschaften messen:
            \begin{itemize}
                \item Ähnlichkeit der gezogenen Maps in kurzem Zeitraum
                \item Wiederholung der selben Map in kurzem Zeitraum
                \item Keine Muster/Nicht-deterministische Verteilung 
                \item Layervotes müssen direkten Einfluss haben 
            \end{itemize}

            Der in diesem Dokument beschriebene Algorithmus soll alle vier genannten Eigenschaften so gut wie möglich Erfüllen.
            Kurz gesagt ist das Ziel des Maprota Generators:\\\
            " Ein probabilistisches System dessen globale Verteilung durch Mapvotes gegeben ist und lokal Wiederholung ähnlicher Maps vermeidet. "
            Anders ausgedrückt:\\
            Das System sollte global einer Verteilung folgen welche die Map/Layervotes wiederspiegeln und lokal eine gewisse Varianz unter den Mapcharakteristiken hervorruft.

            Im weiteren Verlauf wird ein Algorithmus präsentiert welcher alle oben genannten Aspekte so gut wie möglich abdeckt. 
            Es sei aber darauf hingewiesen dass nicht alle Punkte gleichzeitig perfekt umgesetzt werden können aufgrund kritischer Eigenschaften des Systems und der gesetzen Nebenbedingungen.
