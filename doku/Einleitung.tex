\section{Einleitung}
        Squad ist ein Online Multiplayerspiel, welches 2015 im \glqq{}Early Access\grqq{} veröffentlicht wurde.\cite{steampage}
        Für die Umsetzung des Onlinespielens, wird darauf gesetzt, 
        dass Spielserver von Personen oder Gruppen bereitgestellt und verwaltet werden.
        \cite{wiki.serverbrowser}
        Diese Personen/Gruppen sind darum bemüht ihre Server möglichst attraktiv zu halten. \todo{quelle}\\
        Squad wird in Runden gespielt. Eine Runde besteht aus einem vorherrschenden Spielmodus, 
        welcher auf einer Kartenvariation (Layer) gespielt wird. 
        Ein Layer gibt damit an, welcher Spielmodus auf welcher Kartenvariation gespielt wird.\\ \todo{quelle}
        Um die Attraktivität eines Servers zu steigen, wird versucht eine begehrenswerte Reihenfolge an Runden zu spielen.
        Dafür wird für jeden Tag eine Reihenfolge an Runden bestimmt, welche auf dem Server gespielt werden sollen.
        Diese vorgegebene Reihenfolge wird Maprotation oder kurz Maprota genannt.\\
        Dieses Dokument beschäftigt sich mit der Umsetzung eines Programmes, welches eine attraktive Maprota generieren kann.   

        \subsection{Problemdarstellung}
            Was war die Herangehensweise?

        \subsection{Ziel}
            Das fertige Programm soll attraktive Maprotationen erstellen können.
            Dabei müssen die Layervotes mit einbezogen werden. Zudem soll die Maprota, im Bezug auf die Eigenschaften einer Runde, wenig repetitiv sein.
            Außerdem soll eine Karte sich nicht in einem kurzem Zeitraum wiederholen.