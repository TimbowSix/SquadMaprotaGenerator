\documentclass[a4paper, 11pt]{scrreprt}

\usepackage[a4paper, total={6in, 9in}]{geometry}

%========================================== Dokumentinformationen ===================================================

\title{Squad Maprotagenerator 2.0}
\subtitle{Für politische Entscheindungsträger}
\author{timbow, fletschoa, kappakay}

\usepackage[dvipsnames]{xcolor}

\usepackage[
	pdftitle={Squad Maprotagenerator 2.0},
	pdfsubject={},
	pdfauthor={timbow, fletschoa, kappakay},
	pdfkeywords={}
	pdftex=true, 
	colorlinks=true,
 	breaklinks=true,
	citecolor=blue,
	linkcolor=black,	
	menucolor=black,	
	urlcolor=blue
]{hyperref}

%========================================== Klassen ===================================================
\usepackage{nomentbl}
\usepackage{siunitx}
\usepackage{hyperref}
\usepackage{float}
\usepackage{dcolumn}
\usepackage[backend=bibtex,language=german,style=ieee]{biblatex}
\usepackage{todonotes}
\usepackage[]{listofsymbols}
\usepackage{amsmath,amssymb}
\usepackage[german]{babel}
\usepackage{bibgerm}
\usepackage[printonlyused]{acronym}
\usepackage{lmodern}
\usepackage{graphicx}
\usepackage{graphicx, subfigure}
\graphicspath{{img/}}
\usepackage{scrlayer-scrpage}
\usepackage[onehalfspacing]{setspace}
\usepackage{tikz}
\usetikzlibrary{shapes.geometric, arrows}
\usepackage{pdflscape}
\usepackage{multirow}
\usepackage{pgfplots}
\usepackage{pdfpages}
\usepackage{verbatim}


\begin{document}
    
    \maketitle
    
    \tableofcontents
    \newpage


    \newpage
    \listoffigures
    \label{sec:abkuerzungverzeichnis}
    %\addchap{Abkürzungverzeichnis}
    \addcontentsline{toc}{chapter}{Abbildungsverzeichnis}
    \newpage
    %\listoftables
    %\addcontentsline{toc}{chapter}{Tabellenverzeichnis}
    %\newpage

    \begin{acronym}[Rota] %in den [] die abkürzung angeben mit der längsten Beschriftung
        \acro{Rota}{Maprotation}
    \end{acronym}


    \chapter{Übersicht}
    \section{Problemdarstellung}
    \section{Ziel}
    \section{Spezifikationen}
    \section{Metriken}



    \chapter{Aufbau}
    \section{Überliegende Struktur}
    \section{Abhängigkeiten der Generierung}
    \subsection{Biome}
    \subsection{Mapgröße}
    \subsection{Votes}
    \subsection{Modes pro Map}

    \chapter{Tests und Auswertung}

    \chapter{Warum dieser Generator besser ist als Andere}



\end{document}