\documentclass[a4paper, 11pt]{scrartcl}

%\usepackage[a4paper, total={6in, 9in}]{geometry}

%========================================== Dokumentinformationen ===================================================

\title{Squad Maprotagenerator 2.0}
\subtitle{Für politische Entscheindungsträger}
\author{timbow, fletschoa, kappakay}

\usepackage[dvipsnames]{xcolor}

\usepackage[
	pdftitle={Squad Maprotagenerator 2.0},
	pdfsubject={},
	pdfauthor={timbow, fletschoa, kappakay},
	pdfkeywords={}
	pdftex=true, 
	colorlinks=true,
 	breaklinks=true,
	citecolor=blue,
	linkcolor=black,	
	menucolor=black,	
	urlcolor=blue
]{hyperref}

%========================================== Klassen ===================================================
\usepackage{hyperref}
\usepackage[backend=bibtex,language=german,style=ieee]{biblatex}
\usepackage{todonotes}
\usepackage{amsmath,amssymb}
\usepackage[german]{babel}


\usepackage{graphicx, subfigure}
\graphicspath{{img/}}

%include bib data
\bibliography{bibtex.bib}

\begin{document}
    
    \maketitle
    
    \tableofcontents
    \newpage

    Memory Colonel, der du bist in GooseBay, geheiligt werde dein Name.
    Deine Rota komme.
    Deine locktime geschehe.
    Unser täglich Squad gib und heute. Und vergib uns unser Minen legen, wie auch wir vergeben unseren snipern
    Und führe uns nicht in versuchsung, sondern erlöse uns von Tallil
    Denn dein ist die Rota und Biome und die Abwechslung in Ewigkeit.
    Amen

    \newpage
    \listoffigures
    \label{sec:abkuerzungverzeichnis}
    %\addchap{Abkürzungverzeichnis}
    \addcontentsline{toc}{section}{Abbildungsverzeichnis}
    \newpage
    %\listoftables
    %\addcontentsline{toc}{chapter}{Tabellenverzeichnis}
    %\newpage

    
    \input{test.tex}
    \section{Einleitung}
        Squad ist ein Online Multiplayerspiel, welches 2015 im \glqq{}Early Access\grqq{} veröffentlicht wurde.\cite{steampage}
        Für die Umsetzung des Onlinespielens, wird darauf gesetzt, 
        dass Spielserver von Personen oder Gruppen bereitgestellt und verwaltet werden.
        \cite{wiki.serverbrowser}
        Diese Personen/Gruppen sind darum bemüht ihre Server möglichst attraktiv zu halten. \todo{quelle}\\
        Squad wird in Runden gespielt. Eine Runde besteht aus einem vorherrschenden Spielmodus, 
        welcher auf einer Kartenvariation (Layer) gespielt wird. 
        Ein Layer gibt damit an, welcher Spielmodus auf welcher Kartenvariation gespielt wird.\\ \todo{quelle}
        Um die Attraktivität eines Servers zu steigen, wird versucht eine begehrenswerte Reihenfolge an Runden zu spielen.
        Dafür wird für jeden Tag eine Reihenfolge an Runden bestimmt, welche auf dem Server gespielt werden sollen.
        Diese vorgegebene Reihenfolge wird Maprotation oder kurz Maprota genannt.\\
        Dieses Dokument beschäftigt sich mit der Umsetzung eines Programmes, welches eine attraktive Maprota generieren kann.   

    \section{Problemdarstellung}
        Was war die Herangehensweise?

    \section{Ziel}
        Das fertige Programm soll attraktive Maprotationen erstellen können.
        Dabei müssen die Layervotes mit einbezogen werden. Zudem soll die Maprota, im Bezug auf die Eigenschaften einer Runde, wenig repetitiv sein.
        Außerdem soll eine Karte sich nicht in einem kurzem Zeitraum wiederholen.

    \section{Spezifikationen}
    Im folgenden werden unbekannte Begriffe erklärt und/oder definiert.
        \subsection{Allgemeines}
            \subsubsection{Spielmodus}
                Die aktuell (stand Sep. 2022) auf dem {WLS} Server gespielten Spielmodi sind:\\
                \begin{itemize}
                    \item RAAS
                    \item AAS
                    \item Invasion
                    \item TC
                    \item Insurgency
                    \item Destruction
                    \item Skirmish\todo{Tanks ?}
                \end{itemize}
                Die Beschreibung dieser Modi geht über dieses Dokument hinaus.
            \subsubsection{Layer}
                Ein Layer ist einer Karte und einen Spielmodus zugeordnet.
                Pro Runde Squad wird ein Layer gespielt. 
            \subsubsection{Maprotation}
                Eine Maprotation (kurz Maprota) besteht aus einer Liste von vorgegebene Runden, die gespielt werden sollen.
                Jede Runde wird ein Layer gespielt.
            \subsubsection{Clustering}
                Beschreibt die Wiederholung von Squad Runden mit ähnlichen Eigenschaften in einem kurzem zeitintervall.
            \subsection{Attraktive Maprotation}
                Eine attraktive Maprota kann durch ihre Rundenverteilung und Rundenreihenfolge beschrieben werden.
                Gibt es eine positive Korrelation zwischen Layer-Verteilungen einer Maprota und der Layervote-Verteilung ist der Beliebtheitsgrad höher.
                Zudem ist eine Maprota mit kurz aufeinander folgenden, sich stärker unterscheidnenen Maps, beliebter.
                \todo{ist ein bisschen aus der Luft gegriffen}
        \subsection{Eingabe Parameter}
            \subsubsection{Layervote}
                Während einer Runde auf einem WLS Server kann das gespielte Layer, von jedem Spieler,
                positiv oder negativ bewertet werden.
                Diese Stimme einer Person wird Layervote genannt.
            \subsubsection{Biom-Bewertung}
                Eine Karte in Squad ist einem Platz auf der Welt nachempfunden. 
                Für die Einordnung der Karten untereinander werden sie anhand ihrer Eigenschaften bewertet.
                Diese Bewertung wird hier Biom-Bewertung genannt.
        \subsection{Gruppierung der Spielmodi}
                Für eine bessere Übersicht werden hier die Spielmodi Gruppiert.\\
                Casual:
                \begin{itemize}
                    \item RAAS
                    \item AAS
                \end{itemize}
                Intermediates:
                \begin{itemize}
                    \item TC
                    \item Invasion
                \end{itemize}
                Rest:
                \begin{itemize}
                    \item Insurgency
                    \item Destruction \todo{tanks ?}
                \end{itemize}
    
    \section{Metriken}
    Um eine erzeuge Maprota Bewerten zu können, muss sie quantifiziert werden. 
    Dafür werden Metriken definiert welche eine objektive Bewertung, mit gegebenen Daten, zulässt.    



    \section{Überliegende Struktur}
    \section{Abhängigkeiten der Generierung}
    \subsection{Mode}
    \subsection{Biome}
    \subsection{Mapgröße}
    \subsection{Votes}
    \subsection{Modes pro Map}

    \todo{hier mem colonel einfügen}
    \newpage

    \printbibliography[title=Literaturverzeichnis]
    \addcontentsline{toc}{chapter}{Literaturverzeichnis}

\end{document}