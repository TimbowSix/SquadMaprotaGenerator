\section{Theoretische Grundlagen}
    Im Folgenden sollen die zuvor genannten Ziele genau definiert werden. 
    Desweiteren wird ein mathematisches Modell aufgebaut welches den späteren Algorithmus repräsentieren wird.
    \subsection{Definitionen}
    \subsubsection{Maprota}
    Eine Maprota setzt sich aus einer geordneten Menge von einer fest gesetzten Anzahl an Layern zusammen. 
    Es können verschiedene Modi vorhanden sein.
    \subsection{Wahrscheinlichkeitsmaß}
        Wie im vorherigen Kapitel dargelegt ist das Ziel eine Mapverteilung zu bekommen welche einer Vorgegebenen Verteilung folgt. 
        Letztere wird aus den Layervotes generiert.
        Die Maprota kann durch folgende multivariate Verteilung modelliert werden
        \begin{align}
            p_{G,M,L}(g,m,l)    &= P(G=g, M=m, L=l) \\
                                &= P(G=g)\cdot P(M=m|G=g) \cdot P(L=l|G=g, M=m)
        \end{align}
        wobei $G,M,L$ die Zufallsvariablen "Gamemode", "Map" und "Layer" sind und $P(A|B)$ die bedingte Wahrscheinlichkeit für das Ereignis $A$ unter der Nebenbedingungen das zuvor $B$ eingetreten ist.
        Die Wahrscheinlichkeit für ein Layer wie "Yehorivka RAAS v3" ist damit gegeben durch 
        \begin{equation*}
            \mathbb{P}(\text{''Yehorivka RAASv3''}) = p_{G,M,L}(\text{''RAAS'', ''Yehorivka'', ''RAASv3''})
        \end{equation*}
        \subsubsection{Verteilungsfunktionen}
        Die Gamemode Wahrscheinlichkeiten werden von Hand gesetzt. 
        Sollte eine Map $m$ nicht im Gamemode $g$ vorhanden sein, so ist die bedingte Wahrscheinlichkeit $P(M=m|G=g) = 0$. 
        Falls die Map vorhanden ist so wird die Wahrscheinlichkeit zum ziehen der Map errechnet aus den Mapvotes verglichen mit den anderen Maps und der Ähnlichkeit zu den zuvor gezogenen Maps.
        Wenn eine Map $m$ gezogen wurde wird ein Layer $l$ nach der Verteilung der Mapvotes der in frage kommenden Layern gezogen. 
        Es ist dann gegeben durch 
        \begin{equation}
            \mathbb{P}(\text{''Layer A''}) = \frac{S(v_i)}{\mathcal{N}}
        \end{equation}
    \subsection{Oberflächliches Model - Baumdiagramm}
        Die statistische Natur der Maprota kann als "Würfeln" verschiedener Layer verstanden werden. 
        Der Algorithmus setzt sich aus zwei Schichten zusammen: Zunächst wird ein Modus gezogen z.b. "Invasion". 
        Die Wahrscheinlichkeit dass ein Modus gezogen wird ist \textit{a priori} gesetzt und wird als externe Größe in den Generator gegeben.
        Nach ziehen des Modus folgt nun die Auswahl der Map. 
        Die Wahrscheinlichkeit das eine Map gezogen wird hängt von den Mapvotes ab und weiteren internen Parametern. 

        \begin{figure}
            \begin{forest}
                for tree={
                  math content,
                  if n=0{coordinate}{circle,draw=black,fill=red!10},
                  grow=0,
                  l=3cm
                }
                [
                 [\overline{A},edge label={node[midway,below=2pt]{$q_1$}}
                  [\overline{D},edge label={node[midway,below]{$p_{21}$}}
                   [\overline{L},edge label={node[midway,below]{$l_{211}$}}]
                   [L,edge label={node[midway,below]{$l_{212}$}}]
                  ]
                  [D,edge label={node[midway,above]{$p_{22}$}}] 
                 ]
                 [A,edge label={node[midway,above=2pt]{$q_2$}}
                  [\overline{D},edge label={node[midway,below]{$p_{11}$}}]
                  [D,edge label={node[midway,above]{$p_{12}$}}] 
                 ]
                ]
            \end{forest}
        \caption{Baumdiagramm zur Zusammensetzung der Mapwahrscheinlichkeiten. 
        Die Wahrscheinlichkeiten einen Mode zu ziehen sind $q_1$ und $q_2$ für Mode $1$ und $2$ repsektiv.
        Die Wahrscheinlichkeit eine Map unter gegebenen Modus zu ziehen ist gegeben durch $p_{ij}$, d.h. $p_{12}$ ist die bedingte Wahrscheinlichkeit dass Map $2$ gezogen wird wenn vorher Mode $1$ gezogen wurde.}
        \end{figure}
    \subsection{Mapvoteweights}
        Zur Realisierung der Verteilung werden Mapwahrscheinlichkeiten benötigt. 
        Diese werden aus den Layervotes wie folgt gewonnen:\\
        Sei $\mathcal{M}$ die Menge aller Maps. 
        Desweiteren sei $v\in\mathbb{Z}$ die Anzahl an Votes eines Layers.
        Dann wird das Weight für ein $m\in \mathcal{M}$ berechnet nach 
        \begin{align*}
            \mu_i &= \frac{1}{N}\sum_{i=1}^N v_i\\
            w_i &= \exp\left(-(\mu_i-v_i)^2\right)\\
            \bar{w}_i &= \frac{w_i}{\sum_i w_i}\\
            \bar{v}_\text{map} &= \sum_{i=1}^N\bar{w}_i\cdot v_i
        \end{align*}
        Der Mapvote einer Map für einen Modus errechnet sich als das gewichtete arithmetische Mittel aller Votes im selben Modes der Map.
        Das Gewicht eines Votes ist so definiert dass Vote-summen fern des Erwartungswertes weniger stark ins Gewicht fallen.
        Damit soll verhindert werden, dass einzelne "Ausreißer" die globale Map-Bewertung zu sehr herunterziehen.
    \subsection{Die Kugel}
    % was muss hier erklärt werden
    % was ist eine rota
    % Überblick aus dem Game, was sind modes was sind map was sind layer, was sind biome 
    % kleine einführung in mathe 
    % was ist eine gute und eine schlechte rota
    Um den ''Map Charakter'' verschiedener Maps zu vergleichen werden zunächst verschiedene Charakteristiken definiert wie zum Beispiel ''Wüste'' oder ''Stadt''.
    Jede Map wird ein Anteil an jeder Charakteristik zwischen $0$ und $1$ zugeschrieben, was respektiv dafür steht dass die Eigenschaft gar nicht oder absolut zutrifft für die Map.
    Zum Beispiel würde eine Map mit ''Winter-Setting'' welche nur aus Gebirge besteht in jeder der beiden Eigenschaften den Wert $1$ erhalten. 
    Die Bewertung der einzelnen Maps ist jedoch subjektiv.
    Um den Vergleich zweier Maps zu schaffen wird das System wiefolgt repräsentiert:
    Zu jeder Map $m$ existiert ein ''Biom-Vektor'' $\vec{b} = (b_1,.....,b_n)^T$. 
    Hier stehen die Vektorkomponenten $b_i$ für die obengenannte Zuordnung der entsprechenden Charakteristik und $n$ ist die Anzahl der Vorhandenen Charakteristiken.
    Anschließend wird der Vektor normiert sodas $|\vec{b}|=1$.
    Dadurch wird genau genommen wird der Raum aller Maps-Charakteristiken definiert durch die Hyperfläche
    \begin{equation*}
        \mathcal{L} = \{\vec{x}\in\mathbb{R}^n ||\vec{x}|=1 \land 0\leq x_i \leq \smash{\frac{\pi}{2}} \quad \forall i\in\{1...n\}\}
    \end{equation*}
    Das bedeutet das die Maps durch Punkte auf dem Rand der $1/2^n$-tel Kugel beschrieben werden.

    Zwei Maps sind ''ähnlich'' sofern die Punkte nahe beieinander liegen, da in diesem fall die Vektorkomponenten fast alle gleich sind wodurch sich die beiden Vektoren lediglich um einen kleinen Winkel unterscheiden.
    Da die Vektoren normiert sind entspricht dies einer kleinen Distanz auf der Kugeloberfläche.
    Die Distanz $l$ zweier Punkte auf der Kugel wird im Mapweight verrechnet und ist gegeben durch 
    \begin{equation}
        d = 2\arccos(\vec{a}\cdot\vec{b})
    \end{equation}
    für zwei Map-Charakteristiken $\vec{a}$ und $\vec{b}$, wobei $\cdot$ das innere Produkt (''Skalar Produkt'') bezeichnet. 
    \subsection{Mapweight}
    Das Weight errechnet sich als Produkt aus einem ''Distanz-Weight'' und einem ''Mapvote-Weight''. 
    \begin{equation}
        w_m(m,d,v) = \frac{1}{\mathcal{N}}w_d(d,m)w_v(v,m)
    \end{equation}
    wobei $\mathcal{N}$ das Produkt-Weight normiert sodass $\sum_m w_m = 1$.
    \subsubsection{Distanzweight}
        Das Distanzweight ist eine allgemeine kontinuierliche Funktion definiert durch
        \begin{equation}
            w_d : \mathbb{R}^+ \rightarrow \mathbb{R}^+, d \mapsto w_d(d),
        \end{equation}
        und der Nebenbedingung $w_d(d)\overset{d\rightarrow 0}{\longrightarrow}0$.
        In der momentanen Version ist die Funktion gegeben durch
        \begin{equation}
            w_d(d) = 1_{[0,d_\text{min}]}(d)
        \end{equation}
        mit $d_\text{min}\geq 0$ als Mindestdistanz. 
        Sollte eine Map näher als $d_\text{min}$ an einer zuvor gezogenen Map liegen ist das Distanzweight und damit das Mapweight $0$ und wird damit nicht gezogen. 
    \subsubsection{Mapvoteweight}
        Das Mapvoteweight wird aus dem Mapvote berechnet unter Nutzung einer Sigmoid funktion. 
        \begin{equation}
            w_v(v) = \frac{1}{1+\exp\left(-(av+b)\right)}
        \end{equation}
        wobei $v$ die Summe aller Layervotes eines Modus einer Maps ist und $a$ und $b$ zwei freie Parameter sind.
        Während $a$ die Steigung moduliert kann mit $b$ ein Offset erzeugt werden.
        Die Sigmoid funktion erlaubt es ein kontinuierliches weight zu definieren wodurch eine Map oder ein Layer nicht abrupt verschwindet sondern graduierlich ändert.
    \subsection{blub2}
    % was ist eine kugel und was meint tim damit 
    % wie werden die mapvotes ausgerechnet
