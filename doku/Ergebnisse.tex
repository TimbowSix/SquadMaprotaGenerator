\section{Ergebnisse}

    \subsection{Bewertung des Systems}
        \label{sec:bewerten_des_systems}
        Um den entstandenen Maprotagenerator bewerten zu können und den Grad der Qualität festellen zu können,
        werden Metriken zur Hilfe genommen. Die Metriken sind für Squadmaprotas
        allgemeingültig und anhand dessen könnten sie miteinader verglichen werden. Es soll nicht unerwähnt bleiben,
        dass durch eine schlechte oder auch falsche Wahl der Einstellparameter das Maprotasystem leicht bis hin zu
        sehr stark beeinträchtigt werden kann. Genaueres dazu ist unter \ref{s:grenzen_des_systems} nachzulesen.
        Daher ist bei dieser Bewertung zu berücksichtigen, dass von uns wohl überlegte Einstellparameter festgelegt wurden
        und als Referenz für Änderungen herangezogen werden sollten.
        Das Wählen passender Einstellparameter kann im User manual nachgelesen werden.

        Es folgt die Auswertung der Maprota anhand vorgegebenen Einstellungswerten.\\

        \subsubsection{Mapverteilung}
            Die Mapverteilung wird von den Layervotes beeinflusst, dieses ist in Kapitel \ref{sub:AufbauImDetail-Map} nachzulesen.
            Diese Verteilung ist die Vorgabe für das System und es wird eine optimale Annäherung angestrebt. Da durch die
            Zielvorgaben die Verteilungsvorgabe nicht immer erreicht werden kann, tritt eine Abweichung in der Verteilung auf.
            Diese Abweichung wird hier als mittlere quadratische Abweichung (MSD) pro Modus angegeben.
            Für diese Auswertung wurden den Verteilungen genommen, die aus den Layervotes vom 19.09.2022 entstanden sind.
            Dabei ist zu beachten, dass der Modus TC zu diesem Zeitpunkt \glqq{}Verbugged\grqq{} ist und daher
            in der Tabelle \ref{t:Ergebnisse:fehler_Mapverteilung} nicht auftaucht.\\
            \begin{table}[h]
                \centering
                \begin{tabular}{|| c c ||}
                    \hline
                    Modus & MSD \\
                    \hline
                    \hline
                    RAAS & 0.00442 \\ %0.00192
                    \hline
                    AAS & 0.00863 \\ %00090
                    \hline
                    Invasion & 0.00867 \\ %0.00336
                    \hline
                    Insurgency & 0.00924 \\ %0.00836
                    \hline
                    Destruction & 0.01906 \\ %0.04831
                    \hline
                \end{tabular}
                \caption{mittlere quadratische Abweichung Mapverteilung}
                \label{t:Ergebnisse:fehler_Mapverteilung}
            \end{table}

            Um eine Vorstellung für die Zahlen zu Entwickeln wird im Folgenden die angestrebte und generierte Verteilung als
            Diagramm dargestellt (siehe Abbildung \ref{fig:expected_mapverteilung_raas}
            und \ref{fig:generated_mapverteilung_raas}).

            \begin{figure}[htbp]
                \centering
                \includegraphics[width=0.8\textwidth]{RAAS_expected.pdf}
                \caption{Erwartete Mapverteilung im Modus RAAS nach Layervotes vom 19.09.2022}
                \label{fig:expected_mapverteilung_raas}
            \end{figure}

            \begin{figure}[htbp]
                \centering
                \includegraphics[width=0.8\textwidth]{RAAS_generated.pdf}
                \caption{Generierte Mapverteilung im Modus RAAS nach Layervotes vom 19.09.2022 (1.Mio. Layer Rota)}
                \label{fig:generated_mapverteilung_raas}
            \end{figure}

            Bei der Betrachtung der Diagramme
            ist zu beachten, dass es sich hier nur um den Modus RAAS handelt.
            Beispielsweise ist die Karte AlBashrah hier deutlich unterrepräsentiert. Dies ist im Modus Invasion
            nicht der Fall, da die Layervotes dort für AlBashrah deutlich besser ausfallen.
            Zudem lässt sich erkennen, dass die Karte Yehorivka, Blackcoast und Gorodok nicht den angestrebten
            Verteilung erreichen. Dieses Phänomen wird im Abschnitt \ref{s:grenzen_des_systems} näher behandelt.

        \subsubsection{Mode/Modus Verteilung}
            Wie bei den Mapverteilungen kann bei den Modusverteilungen die mittlere quadratische Abweichung als
            quantifizierendes Mitteln genommen werden. Bei den Modi ergibt sich eine MSD von $0.04514$.
            Dieser Wert ist für die vorgesehenen Einstellparameter akzeptabel, da Modi die nicht RAAS oder AAS sind
            einen Mindestabstand haben. In diesem Falle ist dieser Abstand 4 Runden.\\
            % Es ist zu beachten, dass durch Formel \ref{eq:Modeweight} der Ausgangswahrscheinlichkeiten bestimmt werden kann.
        \subsubsection{Varriation der Maps}
            Für den die Messbarkeit, der Differenz aufeinander folgende Maps, kann das arithmetische Mittel der Distanzen
            auf der Hyperfläche genutzt werden. Zudem ist es noch sinnvoll sich den gleitenden Mittelwert der Distanzen
            zu betrachten.\\
            Das arithmetische Mittel der Distanzen beträgt $d_m = 1.08946$\\
            Die Betrachtung des gleitenden Mittelwertes ergibt sich für eine Mittelwertbreite von 5 und einer Rota mit 100000 Layern
            eine Verteilung die auf Abbildung \ref{fig:haufigkeit_gleitender_mittelwert} zu sehen ist.

            \begin{figure}[htbp]
                \centering
                \includegraphics[width=0.9\textwidth]{gleitender_mittelwert_biom_distanz.pdf}
                \caption{Häufigkeiten des gleitenden Mittelwertes der Distanzen}
                \label{fig:haufigkeit_gleitender_mittelwert}
            \end{figure}

            Die auf Abbildung \ref{fig:haufigkeit_gleitender_mittelwert} zusehende Normalverteilung zeigt das sich die höchste
            Häufigkeit zwischen dem eingestellten Minimum (0.4) und dem durch die Kugel gegebenen Maximum ($\pi/2$) befindet.
            Die Lage der Normalverteilung zeigt, dass es ein gutes Maß an Diversität gibt. Zudem ist die
            Verteilung nicht nahe am Maximalrand welches eine Patternbildung begünstigen würde.


        \subsubsection{Patternbildung}
        Um das Auftreten von Patterns in der Rota zu untersuchen wird eine hinreichend große Rota-Historie generiert und anschließend die Häufigkeit aller Patterns betrachtet.
        Hierbei unterscheidet man die Länge $d$ eines Patterns welche durch die Anzahl an Maps innerhalb einer Mapabfolge definiert ist. 
        Das heißt im allgemeinen wird es in jeder Rota $23^d$ verschiedene Patterns geben können bei 23 Maps. 
        Der Ablauf der Patternbildungs-Prüfung ist wiefolg:
        \begin{itemize}
            \item generiere hinreichend viele Tagesrotas mit 30 Maps, ca. 250000 Rotas sind hier ausreichend
            \item für eine feste Länge $d$ suche alle auftretenden Patterns in den Rotas und summiere die Häufigkeiten
            \item Teile die Häufigkeiten für festes $d$ durch die Anzahl an gefunden Patterns 
            \item wiederhole für jedes $d$
        \end{itemize}
        Es sei hier angemerkt dass die Pattern-Auftrittswahrscheinlichkeit $p_\text{pat}\overset{d\rightarrow \infty}{\longrightarrow}0$ und im Regelfall $d\leq5$ ausreichend ist.
        
        \textbf{Definition Patternbildende Rota:}
        Ein Rota Generator wird als \glqq Patternbildend \grqq klassifiziert wenn mindestens ein Pattern innerhalb eines Zeitraums 
        von $T = 30$ Tagen mit einer Wahrscheinlichkeit von $p_0 \geq 5$\% mindestens zweimal auftreten wird. 
        Die Wahrscheinlichkeit für das Auftreten dieses Events $ X_0 $ ist gegeben durch
        \begin{equation}
            \mathbb{P}(X_0) = 1-N p_\text{pat}(1-p_\text{pat})^{N-1}-(1-p_\text{pat})^N
        \end{equation}
        wobei $p_\text{pat}$ die Auftrittswahrscheinlichkeit eines Patterns ist und $N$ die Anzahl an
        gespielten Tagesrotas. Das bedeutet dass eine Maprota als \glqq patternbildend\grqq klassifiziert
        ist wenn mit mehr als $5$\% Wahrscheinlichkeit mindestens ein Pattern doppelt vorkommen
        kann innerhalb eines Monats.
        Zusammengefasst wird folgendes betrachtet 
        \begin{itemize}
            \item berechne für festes $d$ die Auftrittswahrscheinlichkeiten $p_\text{pat}$ aller Patterns
            \item berechne $\mathbb{P}(X_0)$ das Maximale $p_\text{pat}$ 
            \item wenn $\mathbb{P}(X_0)$ unter dem Schwellwert von $5$\% liegt gibt es keine kritische Bildung
            \item wenn $\mathbb{P}(X_0)$ über dem Schwellwert liegt gibt es mindestens ein Pattern was häufiger auftritt
        \end{itemize}
        Im letzten Fall muss anschließend genauer untersucht werden wo die Patternbildung herkommt, wieviele Patterns betroffen sind und ob es Problematisch ist für die Rota. 
        Die Abschätzung setzt ein Idealisiertes System vorraus und ist damit nur als obere Schranke zu verstehen. 
        Das heißt selbst wenn die Auswertung Patterns aufweist kann es trotzdem sein dass diese nicht im Live-System auffällt da Täglich weitaus weniger Layer gespielt werden als in der Theorie angenommen.
        Für die jetzige Implementierung sieht man die Wahrscheinlichkeit zur Auffindung von Patterns in Abbildung \ref{fig:pattern}. 
        Die maximale Wahrscheinlichkeit ist bei Patternlänge drei gegeben durch $p_\text{pat}=0.00105$. 
        Damit ist die Threshold Wahrscheinlichkeit gegeben mit $\mathbb{P}(X_0) = 0.00047=0.047$\% was deutlich unter der kritischen Wahrscheinlichkeit von $5$\% liegt.

        \subsubsection{Map Wiederholung}
            Das nächste und hier letzte benutzte Mittel, um eine Maprota zu bewerten ist, nach wie vielen Runden sich eine Map
            wiederholt. Hierfür wurde eine Histogramm aus einer 100000 Layer Rota erstellt.
            Die Abbildung \ref{fig:haufigkeit_der_map_wiederholung} zeigt die Häufigkeit einer Map Wiederholung. Es ergibt sich ein
            Minimum von 3 Runden bevor sich eine Map wiederholen kann. Am Häufigsten ist jedoch eine Map Wiederholung nach 6 bis 9 Runden.
            Dabei muss bedacht werden, das Squad aktuell (09.2022) 22 spielbare Maps beinhaltet. Daher ist dieses ein gutes Wiederholverhalten.

            \begin{figure}[htbp]
                \centering
                \includegraphics[width=0.9\textwidth]{mapWiederholungsHaufigkeiten.pdf}
                \caption{Häufigkeiten der Map Wiederholung}
                \label{fig:haufigkeit_der_map_wiederholung}
            \end{figure}

        % wie kommt er mit verschiedenen Mapverteilungen klar
        % Was sind die Einstellungen und warum haben wir so gewählt
    \subsection{Grenzen des Systems}
        \label{s:grenzen_des_systems}
        Um diese Sektion am besten nachzuvollziehen zu können, wird empfohlen, erneut einen Blick in das Kapitel \ref{sub:Ziele} Ziele des Systems zu
        werfen. Es wird eine qualitativ hochwertige Maprotation gefordert, die zum Einen der Voteverteilung folgen soll und
        zum Anderen in der Map-Reihenfolge einige gewisse Diversität garantieren soll. Bei genauerer Überlegung ist
        das bereits ein Widerspruch in sich. Angenommen eine einzelne Map hat unendlich viele Stimmen und die Maprota
        folgt strikt den Votes. Es würde darin Resultieren, dass nur noch diese eine Map vorkommen dürfte. Diese, von
        der Maprotation angenommenen Verteilung, bildet aber einen Konflikt mit dem Ziel, dass die Maps eine gewisse Diversität
        bieten sollen. Daher sind an der Stelle die Möglichkeiten das Systems beschränkt und die Voteverteilung kann
        nicht immer voll in einer generierten Rotation abgebildet werden. Maps, die sehr viele Upvotes erhalten, können nur so oft drankommen,
        wie es Distanzweight $w_d$ und der Memory Kernel zulässt. Dieser Aspekt des Systems muss aber nicht als negativer Punkt
        aufgefasst werden, denn niemand möchte dauerhaft nur eine einzige Map spielen (solange es nicht GooseBay ist).
        Dieses \glqq{}Feature\grqq{} wirkt damit aktiv gegen die Befürchtung, welche im Vorfeld angesprochen wurde,
        dass nur noch sehr beliebte Maps wie Yehorivka und Gorodok drankommen.
        Um trotzdem das Optimum zwischen vorgegebener Verteilung und Diversität der Maps zu garantieren wird ein Optimizer
        eingesetzt.\\
        % warum kann man mit gewissen einstellungen die Rota zu zerstören kann
        % optimizer warum das ?
