\section{Einstell Parameter Übersicht}
%\subsection{Config}
\begin{description}
    \item[number\_of\_rotas][int] \newline
        anzahl an Rotas, die generiert werden sollen
    \item[number\_of\_layers][int] \newline
        Wie viele Layer eine Rotation insgesamt beinhalten soll
    \item[seed\_layer][int] \newline
        Wie viele Seed Layer am sich am Start der Rota befinden sollen
    \item[update\_layers][bool] \newline
        Entscheidet, ob beim start layer/votes neu abgerufen werden sollen
    \item[mode\_distribution][dict] \newline
    Struktur: 
    \begin{description}
        \item[pools][dict] \newline
        Beliebige anzahl an Mode pools, die eine beliebige Anzahl an modes beinhalten und deren wahrscheinlichkeit innerhalb des Pools festlegen.
        Wichtig: Es muss immer mindestens den 'main' pool mit mindestens einem Modes geben.
        Struktur:
\begin{lstlisting}[language=json, firstnumber=1]
{
    "pool_name": 
    {
    "Mode1": probability (float),
    "Mode2": probability (float)
    }
}
\end{lstlisting}
        \item[pool\_distribution][dict] \newline
        Wahrscheinlichkeiten für die mode pools, s.o
        Struktur:
\begin{lstlisting}[language=json, firstnumber=1]
{
    "pool_name1": probability (float),
    "pool_name2": probability (float)
}
\end{lstlisting}
    \item[pool\_spacing][int] \newline
    Mindestabstand zwischen \underline{nicht} 'main' pool modes
    \item[space\_main][bool] \newline
    Entscheidet, ob 2x der selbe Mode aus dem Main pool hintereinander kommen dürfen oder sie sich abwechseln müssen.
    \end{description}
\end{description}
\newpage
\textbf{Einstellungen ab hier haben enorme Auswirkungen auf die Generierung und Verteilung.
Änderungen ohne genaueres Verständis der Funktionsweise und des Ablaufs der Generierung nicht empfohlen}
\begin{description}
    \item[biom\_spacing][int] \newline
        Wie lange ein Cluster gelockt wird.
    \item[min\_biom\_distance][float] \newline
        Bioms umkreis Abstand, in dem gelockt wird
    \item[mapvote\_slope][float] \newline
        Slope der Mapvoteweight Sigmoid funktion
    \item[mapvote\_shift][int] \newline
        Shift der Mapvoteweight Sigmoid funktion
    \item[layervote\_slope][float] \newline
        Slope der Layervote Sigmoid funktion
    \item[layervote\_shift][int] \newline
        Shift der Layervote Sigmoid funktion
    \item[use\_vote\_weight][bool] \newline
        Entscheidet, ob die layer nach Votes gewichtet werden sollen \newline
        \textbf{Achtung:} Deaktivieren sorgt \underline{nicht} für eine Gleichverteilung der Maps.
    \item[use\_map\_weight][bool] \newline
        Entscheidet, ob die Maps nach Votes gewichtet werden sollen
    \item[save\_expected\_map\_dist][bool] \newline
        Entscheidet, ob die erwartete Mapverteilung nach Mapvoteweight als Datei gespeichert werden soll
    \item[use\_lock\_time\_modifier][bool] \newline
        Entscheidet, ob die Cluster-Locktime für eine bessere verteilung korrigiert werden darf.
    \item[auto\_optimize][bool] \newline
        Entscheidet, ob bei Änderungen relevanter Werte vor Generierung einer Rota automatisch der Optimizer gestartet werden soll
        Dringend empfohlen.
\end{description}
