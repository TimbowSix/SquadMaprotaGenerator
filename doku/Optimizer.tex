\section{Optimizer}
Der Optimizer ist ein entstandenes U-Boot Projekt, welches gute Parameter für \todo{wo gleichung für ref} die Mapweightgleichungen \todo{ref} findet.
Die Bestimmung optimaler Parameter, sind entscheidend für eine gute Annäherung der generierten Mapverteilung an die Erwartete.
Wahrscheinlich wäre es möglich die Parameter analytisch zu ermitteln. Da dieses aber sehr zeitintensiv sein würde, 
haben wir uns vorübergehend für eine Lösung mit einem einfachen \glqq{}Machine Learning\grqq{} Algorithmus entschieden. 
\subsection{Funktionsweise}
Der Optimizer sucht nach der kleinste Abweichung zwischen der vorgegebene und generierten Mapverteilung. 
Es wird jeweils nur ein Modus betrachtet und optimiert.

Der erste Schritt des Optimizers ist die Variierung des ersten Parameters um ein $\varDelta p$ der Mapweightgleichung.
Anhand der neuen Mapweightgleichung werden neue Mapweights berechnet und eine Maprota generiert.
Die Maprota hat eine ausreichende Größe, das Streungseffekte \todo{das hier ist falsch wie heißt das richtig?} nahezu nicht
mehr vorhanden sind. Nun wird die Abweichung $s_m$ berechnet:
\begin{equation}
    s_m = \sqrt{(d_e[0] - d_g[0])^2 + ... + (d_e[n] - d_g[n])^2}
\end{equation}
mit $n$ $\epsilon$ $\mathbb{Z}$
\todo{tim mach mal mathe in schön bitte}
wobei $d_e$ die diskrete Verteilung der erwarteten Maps sind und $d_g$ die generiert diskrete Verteilung ist.
Im jetzigen Zeitschritt $k$  wird die Abweichung $s_m[k]$ mit der Abweichung $s_m[k-1]$ vergleichen. Bei dem ersten Durchlauf entspricht
$s_m[k-1]$ der Abweichung der Grundverteilung. Sollte die Abweichung $s_m[k]$ kleiner sein als $s_m[k-1]$ bleibt die Variierung um das $\varDelta p$ bestehen. 
Es wird zum nächsten Parameter gewechselt und es wird wieder variiert.
Nachdem ein Minimum mit dem $\varDelta p$ gefunden wurde, wird $\varDelta p$ reduziert 
und die Suche startet neu, bis ein vorgegebenes Minimum für $\varDelta p$ erreicht ist.\\
\subsection{Anwendung}
Das finden neuer Parameter durch den Optimizer, wird nur bei veränderten Layervotes und veränderten Einstellungen benötigt.
Daher wird der Optimizer auch nur bei solchen Veränderungen vor der Generierung automatisch aufgerufen. 
Für eine geringere Laufzeit wird die Optimizer parallel mit jedem Modus ausgeführt.