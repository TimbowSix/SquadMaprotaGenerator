\section{Optimizer}
Der Optimizer ist ein entstandenes U-Boot Projekt, welches gute Parameter für \todo{wo gleichung für ref} die Mapweightgleichungen \todo{ref} findet.
Die Bestimmung optimaler Parameter, sind entscheidend für eine gute Annäherung der generierten Mapverteilung an die Erwartete.
Wahrscheinlich wäre es möglich die Parameter analytisch zu ermitteln. Da dieses aber sehr zeitintensiv sein würde, 
haben wir uns vorübergehend für eine Lösung mit einem einfachen \glqq{}Machine Learning\grqq{} Algorithmus entschieden. 
\subsection{Funktionsweise}
Der Optimizer sucht nach der kleinste Abweichung zwischen der vorgegebene und generierten Mapverteilung. 
Es wird jeweils nur ein Modus betrachtet und optimiert.

Der zu betrachtende Optimizer ist als Maß der Abweichung der ''Soll-Mapverteilung'' und der Mapverteilung des Generators gedacht. 
Es seien $p_i^\text{fi}$ die Mapwahrscheinlichkeiten errechnet nach den Mapvotes und $p_i^\text{gen}$ die Mapwahrscheinlichkeiten nach dem Mapgenerator sodass $\sum_i p_i^\text{fi/gen}=1$.
Da die Generator wahrscheinlichkeiten von den internen Mapweights abhängen ist $p_i^\text{gen}=p_i^\text{gen}(a_{ij})$ eine Funktion der Weight-Koeffizienten. 
Es gilt nun diese so zu Optimieren, dass der Optimizer 
\begin{equation}
    s_m(a_{ij}) = \sqrt{\sum_i \left(p_i^\text{fi}-p_i^\text{gen}(a_{ij})\right)^2}
\end{equation}
minimal wird. 
Das heißt wir versuchen die Gleichung 
\begin{equation}
    \underset{a}{\min}\quad s_m(a_{ij}) = a^*
\end{equation}
zu lösen. 
Hierbei ist aber im Allgemeinen $s_m(a^*_{ij}) \neq 0$ da nicht angenommen werden kann dass ein globales Minimum existiert. 

Der erste Schritt des Optimizers ist die Variierung des ersten Parameters um ein $\varDelta p$.
Dadurch erhält man einen neuen parameter $a_{ij} \rightarrow a_{ij} + \delta a_{ij}$ welcher die generierten Mapwahrscheinlichkeiten ändert.
Die neue Verteilung ersetzt damit die $p_i^\text{gen}$ von zuvor und ein neuer Optimizer Wert wird errechnet. 
Anschließend wird der neue Optimizerwert mit dem alten verglichen. 
Der Shift in $a_{ij}$ wird beibehalten wenn 
\begin{equation}
    s_m(a_{ij}) - s_m(a_{ij}+\delta a_{ij}) > 0 
\end{equation}
erfüllt ist. 
Sollte jeder der 5 Parameter nichtmehr optimierbar sein so wird das $\varDelta p$ verringert.
Das prozedere wird so lange durchgeführt bis $\varDelta p < \varDelta p_\text{min}$ erfüllt ist. 
Die berechneten Koeffizienten $a_{ij}$ werden dann benutzt um die internen Mapweights zu berechnen.
\subsection{Anwendung}
Das finden neuer Parameter durch den Optimizer, wird nur bei veränderten Layervotes und veränderten Einstellungen benötigt.
Daher wird der Optimizer auch nur bei solchen Veränderungen vor der Generierung automatisch aufgerufen. 
Für eine geringere Laufzeit wird die Optimizer parallel mit jedem Modus ausgeführt.